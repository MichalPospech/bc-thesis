
\chapwithtoc{Introduction}


Reinforcement learning has been getting a lot of publicity in recent years, including competing with the best players in DotA 2 \cite{openai2019dota}, playing Texas hold'em poker \cite{Brown885} or mastering Go, chess, shogi and other games \cite{Schrittwieser2020}. Other than that reinforcement learning has been applied to many domains such as robot control \cite{openai2019solving}, energy consupmtion optimisation \cite{LISSA2021100043}, healthcare \cite{yu2020reinforcement} or autonomous driving \cite{kiran2021deep}. However most of the current research is focused on deep reinforcement learning while other potentially promising approaches are mostly ignored. This thesis focuses on exploring one of such approaches, evolutionary strategies, a subclass of evolutionary algorithms.

Evolutionary algorithms are an optimisation metaheuristic inspired by biological evolution. They utilise set of candidate solutions which is being periodically evaluated amd modified using genetic operators in order to improve results in the subsequent generation. They are applied to problems that are hard to solve using conventional methods, such as the traveling salesman problem. \cite{Potvin1996}

The method explored in this thesis is \emph{OpenAI-ES} \cite{salimans2017}. Contrary to classic reinforcement learning approaches it is easy to paralellise, scales very well and does not require differentiable policy. Other method explored is \emph{NS-ES} and its variations \emph{NSR-ES} and \emph{NSRA-ES} \cite{conti2018} which utilise information about the agent's behaviour to direct the search to explore agents which behave in a different manner even at the cost of them potentially performing worse temporarily.

In the first part the task of reinforcement learning is formally introduced using Markov decision processes along with several methods for solving the task. After that, evolutionary algorithms are described and are subsequently extended with evolutionary strategies which are the basis of the methods explored in this thesis. Finally evolutionary strategies applied on reinforcement learning problems are explored along with novelty search. In the following chapter experiments are described along with descriptions of the test environments and their results are shown and discussed. In the penultimate chapter some details regarding technical implementation are shared and explained.