\chapter{Theory}
\label{chap:theory}

\section{Reinforcement learning}
\label{sec:reinf}
\begin{itemize}
    \item problem description (state space, action space, reward...)
    \item various methods \begin{itemize}
        \item value function
        \item criterion of optimality
        \item direct policy search (and various methods)
    \end{itemize}
\end{itemize}
\section{Evolutionary algorithms}
\label{sec:ea}
Evolutionary algorithm (EA) is a subclass of evolutionary computation and belongs to set of general stochastic search algorithm. It is a metaheuristc optimization algorithm that exploits the concept of evolution. Metaheuristics are the higher-level procedures intended to find, produce, or select a lower- level procedures or heuristics which may performs partial search. It is applicable to various optimization problems with limited computation capability and having insufficient or imperfect information. In such situations, it provides adequately good solution.  

An EA is inspired by the mechanism of biological evolution, such as reproduction, mutation, recombination, and selection. A set of candidate solution called population (i.e. elements of function domain) is created randomly to maximize the quality function. Then quality function in the form of abstract fitness function is applied to problem domain. For next generation some better candidates are selected on the basis of fitness function. This is achieved by applying the technique of recombination and/or mutation to them. Recombination is represented by an n-nary operator. This operator can be applied to two or more selected candidates known as parents and it generates one or more new candidates (children) as a result. Whereas mutation is applied to only one candidate and it results in one new child. After executing this recombination or mutation, it generates a set of new candidates on the basis of their fitness function. This is an iterative process. It can be continued until sufficiently good quality candidate is found or for set number of generations. (TODO opsano a mirne editovano, asi ne uplne kosher) \cite{Vikhar2016}
\section{Evolutionary strategies}
\label{sec:es}
TODO \cite{Rudolph2012}
\subsection{CMA-ES}
\label{subsec:cma-es}
TODO \cite{Hansen06}
\section{Evolutionary strategies as replacement for reinfocement learning}
\label{sec:es-reinf}
description what is each paper about, what are the findings and what is the state of the art

