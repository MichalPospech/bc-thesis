\chapter{Theory}
\label{chap:theory}

\section{Reinforcement learning}
\label{sec:reinf}
\begin{itemize}
    \item problem description (state space, action space, reward...)
    \item various methods \begin{itemize}
        \item value function
        \item criterion of optimality
        \item direct policy search (and various methods)
    \end{itemize}
\end{itemize}
\section{Evolutionary algorithms}
\label{sec:ea}
Evolutionary algorithms (EA) are a type of optimisation metaheuristics inspired by the process of bilogical evolution. At first a number of possible solutions to the problem at hand is generated (population) and each solution (individual) is encoded (via a domain-specific encoding) and evaluated giving us the value of its fitness. Then a new population is created using a crossover (combination) of 2 or more individuals which are selected at random with their fitness playing part in the chance to be selected for crossover. Each of the newly created individuals has a chance to be mutated via the mutation operator. Finally a new population is created and enters the next iteration of the EA. The algorithm repeats until the stop condition is met, usually a set number of iterations or small improvement of fitness between 2 generations. \cite{Vikhar2016}
\section{Evolutionary strategies}
\label{sec:es}
\begin{itemize}
    \item abstract definition
    \item + vs. ,
    \item parametrization
    \item guidelines for operators (reachability, unbiasedness, control)
    \item covariances
\end{itemize}
     \cite{Rudolph2012}
\subsection{CMA-ES}
\label{subsec:cma-es}
TODO \cite{Hansen06}
\section{Evolutionary strategies as replacement for reinfocement learning}
\label{sec:es-reinf}
\begin{itemize}
    \item Evolution Strategies as a Scalable Alternative to Reinforcement Learning \cite{salimans2017} \begin{itemize}
        \item algorithm description
        \item comparison with RL
        \item paralellization
        \item smoothing in action vs. param space
    \end{itemize}
    \item Improving Exploration in Evolution Strategies for Deep Reinforcement Learning via a Population of Novelty-Seeking Agents \cite{conti2018} \begin{itemize}
        \item novelty search
        \item ratio of fitness and novelty and its effects
    \end{itemize}
\end{itemize}