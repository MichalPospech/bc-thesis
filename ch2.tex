\chapter{Theory}
\label{chap:theory}

\section{Reinforcement learning}
\label{sec:reinf}
\begin{itemize}
    \item problem description (state space, action space, reward...)
    \item various methods \begin{itemize}
        \item value function
        \item criterion of optimality
        \item direct policy search (and various methods)
    \end{itemize}
\end{itemize}
\section{Evolutionary algorithms}
\label{sec:ea}
Evolutionary algorithms (EA) are a type of optimisation metaheuristics inspired by the process of bilogical evolution. At first a number of possible solutions to the problem at hand is generated (\emph{population}) and each solution (\emph{individual}) is encoded (via a domain-specific encoding) and evaluated giving us the value of its \emph{fitness}. Fitness is a function describing how good that particular individual is and it is everything that is needed for creation of  Then a new population is created using a \emph{crossover} (combination) of 1 or more individuals which are selected using the operator of \emph{parental selection}. Each of the newly created individuals has a chance to be mutated via the \emph{mutation} operator. Finally a new population is selected from \emph{offsprings} and possibly the parents based of fitness and enters the next iteration of the EA and the following generation is chosen using \emph{environmental selection} operator. The algorithm repeats until the stop condition is met, usually a set number of iterations or small improvement of fitness between 2 generations. 

There are many variands of EAs such as genetic algorithms (most common), genetic programming, evolutionary programming, neuroevolution or evolutionary strategies that are further described in following chapter. \cite{Rudolph2012} \cite{Vikhar2016}

\begin{algorithm}
    \begin{algorithmic}[1]
    \caption{Evolutionary algorithm}\label{alg:ea}
        \State initialize population $P^0$ with $n$ individuals
        \State set $t=0$
        \Repeat
            \State $Q^t = \{\}$
            \For{$i \in \{1\dots m\}$}
                \State $p_1,\dots,p_\rho = ParentalSelection(P^t)$
                \State $q = Crossover(p_1,\dots,p_\rho)$ 
                \State $q = Mutation(q)$ with chance $p$
                \State $Q^t = q \cup Q^t$
            \EndFor
            \State $P^{t+1} =EnvironmentalSelection(Q^t\cup  P^t)$
            \State increment $t$
        \Until{stop criterion fullfilled}
    \end{algorithmic}
    \end{algorithm}

\section{Evolutionary strategies}
\label{sec:es}
Evolutionary strategies (ES) are a type of optimisation metaheuristic which further specialises EA and restricts their level of freedom. The selection for crossover is unbiased, mutation is parametrised and thus controllable, individuals which should be put to next generation are chosen ordinally based on fitness and individuals contain not only the problem solution but also control parameters.

More formally ES $(\mu / \rho,\kappa,\lambda)$ has $\mu$ individuals in each generation, which produces $\lambda$ offsprings, each created by crossover of $\rho$ individuals and each individual is able to survive for up to $\kappa$ generations as described in algorithm \ref{alg:es}. This notation further generalizes the old $(\mu,\lambda)$ and $(\mu+\lambda)$ notations, where the "," notation means $\kappa=1$ and "+" notation $\kappa=\infty$. 
\begin{algorithm}
\begin{algorithmic}[1]
\caption{$(\mu / \rho,\kappa,\lambda)$-ES}
\label{alg:es}
    \State initialize population $P^0$ with $\mu$ individuals
    \State set age for each $p\in P^0$ to $1$
    \State set $t=0$
    \Repeat
        \State $Q^t = \{\}$
        \For{$i \in \{1\dots\lambda\}$}
            \State select $\rho$ parents $p_1,\dots,p_\rho \in P^t$ uniformly at random
            \State $q = variation(p_1,\dots,p_\rho)$ with age $0$
            \State $Q^t = q \cup Q^t$
        \EndFor
        \State $P^{t+1} =$ select $\mu$ best (wrt. fitness) individuals from $Q^t\cup \{p \in P^t: age(p)<\kappa\}$
        \State increment age by 1 for each $p \in P^{t+1}$
        \State increment $t$
    \Until{stop criterion fullfilled}
\end{algorithmic}
\end{algorithm}

To design an ES one must first select an appropriate representation for an individual and the most natural one is prefered in most cases, if all parameters are of one type (e.g. a real number) a simple vector will suffice, if the types are mixed, a tuple of vectors is required. This however causes an increased complexity of the variation operator.

As for design of the variation operator there are some guidelines that should be followed when designing it.
\begin{description}
    \item[Reachability] every solution should be reachable from any other solution in a finite number of applications of the variation operator with probability $p > 0$
    \item[Unbiasedness] the operator should not favour any particular subset of solution unless provided with information about problem at hand
    \item[Control] the operator should be parametrised in such way that the size of the distribution can be controlled (practice had shown that decreasing it as the optimal solution is being approached is necessary) 
\end{description}
\todo{kovariance}
\cite{Schwefel1995}
\cite{Rudolph2012}

\subsection{CMA-ES}
\label{subsec:cma-es}
TODO \cite{Hansen06}
\section{Evolutionary strategies as replacement for reinfocement learning}
\label{sec:es-reinf}

Black-box optimisation is an alternative approach to solving RL tasks also known as Direct policy search or neurevolution when applied to neural networks. It has several attractive properties such as indifferenco to distribution of rewards, no need for backpropagation and tolerance of arbitrarily long episodes.

\subsection{OpenAI Evolutionary Strategy}
\label{subsec:openai-es}

Compared to reinfocement learning using evolutionary strategies have the advantage of not needing a gradient of the policy performance. Also as the state transition function is not known  the gradient can't be computed using backpropagation-like algorithm. Thus some noise needs to be added to make the problem smooth and the gradient to be estimable. Here is where reinfocement learning and evolutionary strategies differ, reinfocement learning adds noise in the action space (actions are chosen from a distribution) while evolutionary strategies add noise in the parameter space (parameters perturbed while actions are deterministic).



Not requiring backpropagation has several advantages over other RL methods. First the amount of computation necessary for one episode of ES is much lower (about one third, potentially even less for memory usage). Not calculating gradient using analytic methods also protects these methods from suffering from \emph{exploding gradient} which is a common issue with recurrent neural networks. And last, the network can contain elements that are not differentiable such as hard attention. 

As the ES could be seen as method for computing a derivative estimate using finite differences in randomly chosen direction it would suggest that it would scale poorly with dimensions of parameters $\theta$ same as the finite differences method. In theory the number of necessary optimisation steps should scale linearly with the dimension. That however doesn't mean that larger networks optimised using ES will perform worse than smaller ones, that depends on the difficulty (intrinsic dimension) of the problem. The network will perform the same however it will take more optimisation steps to do so. 

In practice ES performs slightly better on larger networks and it is hypothesised that it is for the same reason as why it is easier to optimise large networks using standard gradient based methods: larger networks have fewer local minima. 

Due to perturbing the parameters and not the actions ES are invariant to the frequency at which the agent acts in the envirionment. Tradtional MDP-based reinforcement learning methods rely on \emph{frameskip} as one their parameters that is crucial to get right for the optimization to be successful. While this is solvable for problems that do not require long term planning and actions, long term strategic behaviour poses a challenge and reinfocement learning needs hiearchy to be succesful unlike evolutionary strategy.


\subsection{Novelty search}

While the main drive in of improvement in ES is the value of fitness (how "good" the result is), novelty search takes a different approach. Novelty search is focused on finding different solutions, as it is inspired by nature's drive towards diversity. Each policy has its novelty calculated with respect to previous policies and search is directed to parts of search space with high novelty. This approach makes it less succeptible to local optima created by deceptive rewards than reward-based method. 

Each policy $\pi$ gets assigned its domain-dependent behavorial characteristics $b(\pi)$ (e.g. final position of the agent) and it is added to an archive set $A$ of characteristics of previous policies. Then the novelty $N(b(\pi), A)$ is calculated as average distance from $k$ nearest neighbours from the archive set $A$.

\subsubsection{Combination with evolution strategies}

To find and follow the gradient of expected novelty with respected to $\theta^t$ we use the framework outlined in \ref{subsec:openai-es}. 

\todo{Calculation of gradient}

It is possible because archive $A$ is fixed during one iteration and is updated only at the end. Only characteristics corresponding to each $\theta^t$ are added to $A$, as adding each sampled would cause the archive $A$ to inflate too much increasing the complexity of calculation of nearest-neighbours.

To encourage additional diversity an initial meta-population of $M$, selection of $M$ is domain dependent, agents is created. While it is possible to optimise the behaviour of a single agent and reward it for behaving differently than its ancestors, this way we get the benefits of population-based exploration. \todo{Describe?} Each agent has parameters $\theta^m$ and is being rewarded for behaviour different from all prior agents, thus we get $M$ differently behaving policies.

$M$ random parameter vectors are initialised and in each iteration one is selected to be updated. The selection probability is proportional to its novelty.\todo{Selection prob.}

To perform the update step, we need to calculate the gradient estimate of expected novelty with respect to $\theta^m_t$ \todo{Equation}. After updating the individual, a new behavorial characteristics $b(\pi_{\theta^m_{t-1}})$ is calculated and added to the archive $A$. 

This process is repeated for a predetermined number of times as novelty search is not supposed to converge to a "best" solution and returns the best performing policy which is being preserved during the run of the algorithm.

\begin{itemize}
    \item Evolution Strategies as a Scalable Alternative to Reinforcement Learning \cite{salimans2017} \begin{itemize}
        \item algorithm description
        \item comparison with RL
        \item paralellization
    \end{itemize}
    \item Improving Exploration in Evolution Strategies for Deep Reinforcement Learning via a Population of Novelty-Seeking Agents \cite{conti2018} \begin{itemize}
        \item directed exploration
        \item novelty search
        \item algorithms description
        \item ratio of fitness and novelty and its effects
        
    \end{itemize}
\end{itemize}